%%; This is `eptrip.tex' for e-pTeX v3.141592653-2.6 as of Jan 28, 2021.
%%;
%%; This is a diabolical test file for e-pTeX, an extension of TeX82.
%%; It is not as diabolical as `trip.tex', Knuth's torture test for TeX.
%%;
%%; ==>> Do not use this TeX code and its macros as an example  <<==
%%; ==>> how you can make use of the new e-TeX features!  It is <<==
%%; ==>> meant to test these extensions and often uses slightly <<==
%%; ==>> wrong e-TeX input to produce errors and warnings.      <<==
%%;
\catcode`\{=1\catcode`\}=2\catcode`\#=6 \let\bgroup={ \let\egroup=}
\def\etripdate   {2021-01-28}
\def\texversion  {3.141592653}
\def\etripversion{2.6}
%
\message{This is the e-Trip test [\etripdate] for
         e-TeX v\texversion-\etripversion.}
%
% --- Make sure that e-Trip is run in e-TeX extended mode
\def\stop#1{\message{Emergency stop: #1!}}
\expandafter\ifx\csname eTeXversion\endcsname\relax
  \stop{You aren't using e-TeX in extended mode}
  \message{(Do not forget to give an asterisk `*' as the first non-blank}
  \message{character to make e-IniTeX enter extended mode.)}
  \expandafter\end
\fi
%
% --- Check for e-TeX version
\def\1.#1#2\relax{\bgroup
  \edef\1{\egroup
    \def\noexpand\2{\number\eTeXversion\eTeXrevision}%
    \def\noexpand\1{\number\eTeXversion.#1}}\1}
\expandafter\1\eTeXrevision\relax
\message{(You are using e-TeX version/revision \2)}
\ifx\1\etripversion \else \stop{I have expected e-TeX
     v\texversion-\etripversion...}\expandafter\end\fi
\let\1=\5 \let\2=\5
%
% --- a special branch for e-IniTeX
\ifx\einitex\undefined \def\einitex{}
  %
  \message{e-IniTeX: Assigning category codes,}
  \catcode`\$=3\catcode`\&=4
  \catcode`\^=7\catcode`\^^I=10\catcode`\_=8
  \message{tracing switches,}
  \tracingstats=4\tracinglostchars=2
  \message{other codes,}
  \endlinechar=`\^^M \newlinechar=`\^^J
  %
  \message{definitions for e-VirTeX e-Trip test run,}
  \def\error{\immediate\write15{Bug in your e-TeX implementation!}%
    \immediate\write15 }
  \def\typeout{\immediate\write15 }
  \def\empty{} \def\space{ }
  %
  \message{constants and registers,}
  \chardef\zero=0\chardef\one=1\chardef\two=2
  \countdef\ctmp=255 \countdef\cndx=254
  %
  \message{fonts,}
  \fontdimen12\nullfont=13pt
  \font\trip=etrip \hyphenchar\trip=`1 \trip
  \textfont0=\trip \textfont1=\trip
  \font\smalltrip=etrip scaled 500\relax % our symbols font
  \fontdimen22\smalltrip=7pt
  \textfont2=\smalltrip \scriptfont2=\smalltrip
  \scriptscriptfont2=\smalltrip
  \font\bigtrip=etrip at 20pt\relax % our extension font
  \textfont3=\bigtrip \scriptfont3=\bigtrip
  \scriptscriptfont3=\bigtrip
  %
  \message{some math characters,}
  \delcode`\[="161361 % small (family 1, character "61 (a)), large (3,"61)
  \delcode`\|="142342 % small (family 1, character "42 (B)), large (3,"42)
  \delcode`\]="162362 % small (family 1, character "62 (b)), large (3,"62)
  %
  \message{some math parameters,}
  \thinmuskip=18mu plus 3.6mu
  \medmuskip=27mu plus 9mu minus 18mu
  \thickmuskip=36mu minus 7.2mu
  %
  \message{hyphenation,}
  \lefthyphenmin=2\righthyphenmin=2
  \begingroup
    \def\x{\patterns{%
      .pp1aqq. .up1aqq. .ppb1qq. .upb1qq. .pp1r1qq. .up1r1qq.}}
    \language=0 \x \savinghyphcodes=1
    \language=1 \lccode`A=`a \lccode`B=`b \x
    \language=2 \lccode`A=`r \lccode`B=`b \x
    \language=3 \lccode`A=`a \lccode`B=`r \x
    {\def\1{\lccode\count20=0 \ifnum\count20<255 \advance\count20 1
        \expandafter\1\fi}\count20=0 \1
      \language=4 \patterns{}}
    {\lccode`B=`b \hyphenation{qqB-pp}}
    \count20=\interactionmode \nonstopmode
    {\setbox0\vbox{\parfillskip=0pt
      \hbadness=0 \showboxdepth=0
      \hsize=16383.99999pt \pretolerance=-1 \tolerance=-1
      \trip\ ppaqq upaqq ppbqq upbqq pprqq uprqq qqbpp}}
    {\lccode`B=`b \language=4 \hyphenation{-q-}}
    \interactionmode=\count20
    {\lccode`B=`b \hyphenation{qq-B-pp}}
  \endgroup
  %
  \message{enable e-TeX enhancements (TeXXeT),}
  \TeXXeTstate=1
  %
  \message{prepare saved items (not to be dumped),}
  \bgroup \savingvdiscards=1
    \vfill \penalty 1234 % set \pagediscards
    \setbox0=\vbox{\vbox to10pt{}\vskip5pt\penalty-4321}
    \setbox1=\vsplit0 to10pt % set \splitdiscards
  \egroup
  %
  \message{everyjob ...dumped.}
  \everyjob={\message{e-IniTeX: e-Trip format loaded.}}
  \expandafter\dump
\fi

%
% -- Test \lostchars=1 and 2
\begingroup
  \tracingonline=0
  \setbox0=\hbox{%
    \tracinglostchars=0 \nullfont a%
    \tracinglostchars=1 \nullfont b%
    \tracinglostchars=2 \nullfont c%
  }
\endgroup
%
% -- Do the rest in silent, almost all of it:
\batchmode

%
% -- Check that all e-TeX enhancements are switched off
\ifnum\TeXXeTstate=0 \else
  \error{dumped e-TeX enhancement state registers aren't switch off}
\fi

%
% ============================== input/output: new csnames
%
% -- Check table of equivalents and the routines |id_lookup|,
%    |print_cmd_chr| for all new e-TeX control sequences
% -- Check new primitive \unexpanded (first test)
\typeout{Checking input/output of new csnames:}
\begingroup
  \edef\1{%
    \unexpanded{%
% --- additional e-TeX V 1 primitives
      \eTeXversion \eTeXrevision \showgroups \showtokens
      \tracingassigns \tracinggroups \tracingifs \tracingscantokens
      \currentgrouplevel \currentgrouptype \middle \lastnodetype
      \TeXXeTstate \beginL \endL \beginR \endR \predisplaydirection
      \marks \topmarks \firstmarks \botmarks
      \splitfirstmarks \splittopmarks
      \protected \unexpanded \detokenize \scantokens \readline
      \unless \ifdefined \ifcsname \everyeof \interactionmode
% --- additional e-TeX V 2 primitives
      \currentiflevel \currentiftype \currentifbranch \showifs
      \fontcharwd \fontcharht \fontchardp \fontcharic \iffontchar
      \tracingnesting \parshapelength \parshapeindent \parshapedimen
      \numexpr \dimexpr \glueexpr \muexpr \mutoglue \gluetomu
      \gluestretchorder \glueshrinkorder \gluestretch \glueshrink
      \savingvdiscards \pagediscards \splitdiscards
      \lastlinefit \savinghyphcodes \interlinepenalties
      \clubpenalties \widowpenalties \displaywidowpenalties
      }%
    }

% loop through the list, count the number of csnames,
% insert the \newlinechar after the 2nd, 4th, 6th, ...
  \ctmp=0\def\3{}
  \def\2#1{\ifx\relax#1\else
    \advance\ctmp\one
    \edef\3{\unexpanded\expandafter{\3 #1}\ifodd\ctmp\else ^^J\fi}%
    \expandafter\2\fi}
  \expandafter\2\1\relax
  \typeout{e-TeX Version \number\eTeXversion\space has
    \number\ctmp\space new csnames:}%
  \show\3\endgroup

%
% ============================== create and output nodes
%
% -- Check |new_...| routines and |short_display|, |show_node_list|
%    for all new e-TeX node types and subtypes
% -- Check for new primitives disabled when read.
\typeout{Checking creation and printing for new node types:}
\tracingonline=1
\showboxbreadth=255\showboxdepth=255
\begingroup
  \setbox0=\vbox{\hsize=0pt\TeXXeTstate=1\relax
    %% []\trip a[]b[]p$q [] t$u[][]    % <== \endR\endL at end
    a\beginL b\beginR p$q\left[\mathpunct{r}\middle|s\right]t$u
    %% \trip aMb
    $$aMb$$    % \hbox ..., display
    %% [][]\trip p[]q$st$u[]    % <== \beginL\beginR at beginning
    p\endR q\mathsurround=12.3pt$st$u    % <== \endL at end
    %% \trip rMs
    $$rMs$$    % \hbox ..., display
    %% []\trip tp[]u    % <== \beginL at beginning
    tp\endL u
    %% \trip rMs
    $$rMs$$    % \hbox ..., display
    %% \trip tp    % <== no \mark(s), no \(begin/end)(L/R)!
    tp\mark{0old}\marks0{0}\marks1{1}\marks15{15}\marks65535{65535}%
    \marks-1{-1}\marks65536{65536}% Bad register (-1) ...and (65536).
    \TeXXeTstate=0\relax
    \beginL\beginR\endL\endR % Improper \(begin/end)(L/R)
  }\showbox0
\endgroup

%
%
% ============================== \interactionmode
%
% -- Check new special register \interactionmode
\typeout{Checking \string\interactionmode:}
\begingroup \ctmp=\interactionmode % save current value
\nonstopmode
% test inquiry using \interactionmode
\def\3{\typeout{current interactionmode (l.\number\inputlineno):
  \ifcase\interactionmode batch\or nonstop\or scroll\or
     errorstop\else UNKNOWN!\fi}}
\def\1#1{\relax\ifnum#1=\interactionmode \3 \else
  \typeout{wrong interactionmode:
           \number\interactionmode\space should be \number#1!}\fi}
\batchmode     \1 0
\nonstopmode   \1 1
\scrollmode    \1 2
\errorstopmode \1 3
% test setting |interaction_mode| using this special register
\def\2#1{\interactionmode=#1\relax \1{#1}}
  \nonstopmode
  % generate two errors:
  \interactionmode=-1 \1 1 % nonstop
  \interactionmode=4  \1 1 % nonstop
\2 2 \2 2 \2 1 \2 1 \2 3 \2 3 \2 0 \2 0
\1 0{\interactionmode=3}\1 3 % global assignment!
% Test for the correct call of |new_interaction|, i.e.
% correct setting of |selector|:
\interactionmode=1 \message{1 (l.\number\inputlineno)}% log+term
\interactionmode=0 \message{0 (l.\number\inputlineno)}% log only
\interactionmode=1 \message{1 (l.\number\inputlineno)}% log+term
% Some tests with TeX's old command and register assignments
\batchmode \1 0 \2 3 \scrollmode \1 2 \2 1
\2 \ctmp
\endgroup

%
% -- Check \tracingifs
\typeout{Checking \string\tracingifs:}
\begingroup
  \def\1#1\1{\def\2{#1\iffalse \else \fi}#1\iftrue \2\else \2\fi}
  \tracingifs=1
  \1\1
  \1\unless\1
  \tracingcommands=2
  \ifdefined\hbox \unless\ifdefined\abc \ifcase 2 \or \or
    \ifcsname hbox\endcsname \ifcsname abc\endcsname \abc \else
      \ifx\abc\relax \else \ifx\abc\relax \fi \fi \fi \fi \or\fi \fi \fi
\endgroup
%
% -- Check \tracingassigns
\typeout{Checking \string\tracingassigns:}
\begingroup
  \tracingrestores=1 \tracingassigns=1
  \global\font\6=etrip at 11pt \font\6=etrip at 11pt 
  \global\count17=7 \count17=7
  \global\def\9{\relax} \global\let\8=\9 \let\8=\9
  \bgroup
    \global\font\5=etrip at 12pt \font\5=etrip at 12pt \let\4=\5
    \global\count17=0 \count17=0
    \global\let\9=\7 \global\let\8=\9 \let\8=\9
  \egroup
\endgroup

%
% -- Check \currentgrouptype, \currentgrouplevel, and \lastnodetype
\typeout{Checking \string\currentgrouptype,
   \string\currentgrouplevel, and \string\lastnodetype:}
\typeout{current group level (l.\number\inputlineno):
  \ifcase\currentgrouplevel outer level\else BAD!\fi}
\typeout{current group type (l.\number\inputlineno):
  \ifcase\currentgrouptype bottom level\else BAD!\fi}
\begingroup
  \def\4{\typeout{current group type (l.\number\inputlineno):
    \ifcase\currentgrouptype bottom level\or simple group\or
      hbox group\or adjusted hbox group\or vbox group\or
      vtop group\or align group\or no align group\or
      output group\or math group\or disc group\or
      insert group\or vcenter group\or math choice group\or
      semi simple group\or math shift group\or
      math left group\else UNKNOWN!\fi}}
  \def\1 #1 {\relax\ifnum#1=\currentgrouptype \4\else
    \typeout{wrong current group type:
             \number\currentgrouptype\space should be #1!}\fi}
  \def\5{\edef\6{\ifcase\lastnodetype char node\or hlist node\or
      vlist node\or rule node\or ins node\or mark node\or
      adjust node\or ligature node\or disc node\or whatsit node\or
      math node\or glue node\or kern node\or penalty node\or
      unset node\or math mode node\else
      \ifnum-1=\lastnodetype empty\else UNKNOWN!\fi\fi}%
    \typeout{last node type (l.\number\inputlineno): \6}}
  \def\2 #1 {\relax\ifnum#1=\lastnodetype \5\else
    \edef\6{\number\lastnodetype}%
    \typeout{wrong last node type: \6 should be #1!}\fi}
  \def\3 #1 {\relax\ifnum#1=\currentgrouplevel \else
    \typeout{wrong current group level:
             \number\currentgrouplevel\space should be #1!}\fi}
  \tracinggroups=1
  \1 14 % semi simple group
  \3 1 % group level 1
  \setbox0=\vbox{\1 4 % vbox group
    \hbox{\1 3 % adjusted hbox group
      {\1 1 % simple group
        }\2 -1 % empty list
      \hbox{\1 2 % hbox group
        \discretionary{\1 10 }{\1 10 }{\1 10 % disc group
          \3 5 % group level 5
          }\2 8 % disc node
        A\2 0 % char node
        AA\2 7 % ligature node
        \insert27{\1 11 % insert group
          }\2 4 % ins node
        \vadjust{\1 11 % insert group
          }\2 6 % adjust node
        \mark{}\2 5 % mark node
        \vrule \2 3 % rule node
        \hfil \2 11 % glue node
        \kern 0pt\2 12 % kern node
        \penalty 0\2 13 % penalty node
        \vtop{\1 5 % vtop group
          }\2 2 % vlist node
        \write5{}\2 9 % whatsit node
        $\1 15 % math shift group
          {\1 9 % math group
            }
          $\2 10 % math node
        \TeXXeTstate=1\beginR\2 10 % math node
        \endR\2 10 % math node
        \valign{#\cr\1 6 % two align groups
          \cr\noalign{\1 7 % no align agroup
            \2 14 % unset node
            }
          }
        }
      }\2 1 % hlist node
    $$\1 15 % math shift group
      \2 -1 % empty list
      \mathchoice{\1 13 }{\1 13 }{\1 13 }{\1 13 % math choice group
        \vcenter{\1 12 % vcenter group
          }\2 15 % math mode node
        }\2 15 % math mode node
      \left.\1 16 % math left group
      \middle.\1 16 % math left group
      \middle.\1 16 % math left group
      \right.
      $$
    }
  \output={\1 8 % output group
    \setbox0=\box255
    \setbox0=\vbox{ % vbox group % this is definitely weird!
      \hbox spread 5pt{\hfil % adjusted hbox group
        \raise5pt\hbox{ % hbox group
          { %simple group
            \valign{#\cr % two align groups
              \noalign{ % no align group
                $ % math shift group
                  \vtop{ % vtop group
                    $$ % math shift group
                      { % math group
                        \left.\middle. % math left group
                          \mathchoice{}{}{ % math choice group
                            \vcenter to 7pt{\vss % vcenter group
                              \noindent\vadjust{ % insert group
                                \discretionary{}{% disc group
                                  \showgroups % show all of them
                                  }{}
                                }
                              }
                            }{}
                          \right.
                        }
                      $$
                    }
                  $
                }
              }
            }
          }
        }
      }
    }
  \hbox{}\vfil\penalty-10000
  \deadcycles=0
\endgroup

%
% -- Check expansion of \protected macros
\typeout{Checking expansion of \string\protected\space macros:}
\begingroup
  \protected\def\2{} \show\2
  \let\3=\2 \show\3
  \protected\unexpanded\bgroup\2\protected\3\protected\def\1{\2}} \show\1
  \tracingmacros=1
  \message{\expandafter\1\1}
  \typeout{\expandafter\1\1}
  \setbox0=\vbox{\special{\expandafter\1\1}} \showbox0
  \edef\5{\expandafter\1\1} \show\5
  \setbox0=\vbox{\halign{#\1&#\span\1\cr
    \protected\def\1{\3}&\protected\def\1{\3}\cr}}
\endgroup

%
% -- Check \scantokens, \tracingscantokens, \readline, and \everyeof
\typeout{Checking \string\scantokens, \string\tracingscantokens,
  \string\readline, and \string\everyeof:}
\begingroup
  \newlinechar=`\^^J
  \def\1{\endgroup ^^J\fi ^^J\bgroup ^^J\iffalse \else}
  \tracingoutput=1
  \shipout\vbox{\global\advance\count\zero by\one
    \openout1=\jobname.out
    \write1{\unexpanded\expandafter{\1}}
    \closeout1}
  \begingroup \iftrue \input\jobname.out \egroup \fi
  \begingroup \iftrue \scantokens\expandafter{\1} \egroup \fi
  \errorcontextlines=1000 \tracingscantokens=1
  \begingroup
    \tracingnesting=1
    \begingroup \iftrue \scantokens\expandafter{\1} \egroup \fi
      \tracingnesting=2
      \begingroup \unless\iffalse \scantokens\expandafter{\1} \egroup \fi
% and now a really weird (although legitimate) combination
      \setbox0=\hbox\bgroup
      \unless\iftrue\else
      \scantokens{\hbox\bgroup^^J\ifcase0^^J\tracingscantokens=0^^J
        \newlinechar=`\^^Z
        \scantokens{\egroup^^Z\else\fi^^Z\def\1{\egroup\fi}^^Z\1^^Z^^J
          \setbox0=\vtop\bgroup^^Z\ifnum0=0^^Z$$^^Z\ifinner\else^^Z
          \csname iffalse\endcsname^^J
          }^^J % end of inner scantokens
        $$^^J\fi
        } % end of outer \scantokens
      \egroup
      \else\fi
    \endgroup
  \begingroup \iftrue \let\9=\endgroup \def\endgroup{\9\9}
    \input\jobname.out \egroup \fi
  \begingroup \iftrue \let\9=\endgroup \def\endgroup{\9\9}
    \scantokens\expandafter{\1} \egroup \fi

  \edef\1{\scantokens{\begingroup} % <== error
  \everyeof={\noexpand}
  \edef\1{\scantokens{\begingroup}\endgroup} % <== OK
  \errorcontextlines=100
  \def\2{\begingroup\scantokens{\message{level=\the\currentgrouplevel}}%
    \endgroup}
  \bgroup\bgroup\bgroup \edef\1{\egroup\egroup\egroup
    \everyeof={\noexpand\ifnum\number\currentgrouplevel
      \unexpanded{>\currentgrouplevel\2\else\9\fi}}}\1\2

  \openin7=\jobname.out
  \bgroup
    \def\1{\unless\ifeof7\readline7to\2%
      \endlinechar=-\one\2\expandafter\1\fi}
    \tracingcommands=2 \tracingrestores=1 \tracingassigns=1
    \setbox25=\hbox{\1\showlists}
    \egroup

\endgroup

%
% -- Check \marks etc., \showtokens, and \detokenize
\typeout{Checking \string\marks\space etc., \string\showtokens, and
  \string\detokenize:}
\begingroup
  \newlinechar=`\^^J
  \def\4{\message{Current marks:\9^^J}}
  \def\3#1{\edef\9{\9^^J \detokenize{#1}=\detokenize\expandafter{#1}.}}
  \def\2#1 {\3{\splitfirstmarks#1}\3{\splitbotmarks#1}}
  \def\0{\let\9=\empty}
  \def\7{\penalty0\vskip40pt}
  \def\1{\0\20 \21 \23 \4}
  \setbox0=\vbox{%
    \marks0{0-1}\marks1{1-1}\vskip1pt\vbox to9pt{}
    \mark{0-2}\marks3{}\hbox{}\vskip5pt\vfil\7
    \marks0{0-3}\vskip1pt\vbox to19pt{}\marks3{3-3}\vskip3ptplus1fil\7
    \vbox to30pt{}\vfil\7\hbox to10pt{}}
  \setbox4=\copy0
  \1 \showbox0 \let\5=\3 \def\3#1{\showtokens\expandafter{#1}\5{#1}}
  \setbox1=\vsplit0to17pt
  \1 \showbox1 \showbox0 \let\3=\5
  \setbox1=\vsplit0to27pt
  \1 \showbox1 \showbox0
  \setbox1=\vsplit0to37pt
  \1 \showbox1 \showbox0
  \def\2#1 {\3{\topmarks#1}\3{\firstmarks#1}\3{\botmarks#1}}
  \vsize=35pt \output{\1 \showbox255 \setbox0=\box255}\1
  \unvbox4\marks0{}\marks1{}\marks3{}\vfil\penalty-10000
  \hbox{}\marks0{}\marks1{}\marks3{}\vfil\penalty-10000
  \hbox{}\vfil\penalty-10000 % reclaim sparse array memory!
  \deadcycles=0
\endgroup

%
% -- Check \middle
\typeout{Checking \string\middle:}
\begingroup
  \scriptfont1=\textfont1 \scriptscriptfont1=\scriptfont1
  \setbox0=\vbox{\middle \par \right \par} % <== 8 errors
  \setbox0=\hbox{%
    $
      \left[p
        \left[q
        \over r
          \left[p
          \middle|q
          \middle|q
            \left[p
            \middle|q
            \middle|q
            \over r\showlists
            \right]\showlists
          \right]\showlists
        \right]\showlists
      \right]\showlists
    $\showlists
    }
  \setbox0=\hbox{$\displaystyle
    \left[\scriptstyle\middle|\mathchoice{p}{q}{r}{s}
    \raise16.5pt\hbox{}\right]$\showlists}
  \setbox0=\hbox{$\scriptscriptstyle
    \left[\displaystyle\lower3pt\hbox{}\middle|\mathchoice{p}{q}{r}{s}
    \right]$\showlists}
\endgroup

%
% -- Check TeX--XeT's new primitives
\typeout{Checking TeX--XeT enhancements:}
\begingroup
% -- incorrect matching & conversion
% (extra `endL/R' nodes are converted to `kern 0.0' nodes)
  \setbox0=\vbox{\TeXXeTstate=1
    \parfillskip=0pt plus1fil\hsize=10pt\parindent=.5pt
    \beginL\kern1pt\par
    \beginR\kern2pt\par
    \beginL\beginR\kern3pt\penalty0\hbox{}\kern3pt\endL\endR\endL\par
    \endL\kern4pt\endR\par}
  \showbox0
% %%%%%%%%%%%%%%%%%%%%%%%%%%%%%%%%%%%%%%%% STILL INCOMPLETE!!!
% %%%%% Missing: Test for functionality of all new primitives
%   \TeXXeTstate \beginL \endL \beginR \endR \predisplaydirection
% %%%%%
\endgroup

%
% -- Check optimized \aftergroup
\typeout{Checking optimized \string\aftergroup:}
\begingroup
  \errorcontextlines=1000
  {{\aftergroup\x\aftergroup}\aftergroup{\aftergroup\relax}}
\endgroup

%
% -- Check \showifs, \currentiftype, \currentiflevel, and \currentifbranch
\typeout{Checking \string\showifs, \string\currentiftype,
  \string\currentiflevel, and \string\currentifbranch:}
\begingroup
  \def\1{ !BAD} \edef\2{ \ifnum0=\currentifbranch OK\else!BAD\fi}
  \iftrue \expandafter \ifnum \number\currentifbranch=1
    \iffalse \else \expandafter \ifnum \number\currentifbranch=-1
      \ifnum \currentifbranch=0 \let\1=\2 \fi \fi \fi \fi \fi
  \message{\string\currentifbranch \1}
  \def\1 #1 {\edef\2{\number\currentiflevel}%
    \ifnum\2=#1
      \typeout{current if level (l.\number\inputlineno): \number\2}%
    \else
      \typeout{wrong current if level: \2\space should be #1!}%
    \fi}
  \1 0 % if level 0
  \iftrue \1 1 % if level 1
    \iftrue \1 2 % if level 2
    \fi
  \fi
  \def\1 #1 #2 {\edef\2{\number\currentiftype}%
    \edef\3{\number\currentifbranch}%
    \ifnum\2=#1 \4\else
      \typeout{wrong current if type: \2\space should be #1!}\fi
    \ifnum\3=#2 \5\else
      \typeout{wrong current if branch: \3\space should be #2!}\fi}
  \def\4{\typeout{current if type (l.\number\inputlineno):
    \ifnum\2<0 \string\unless\6{-\2}\else\6{\2}\fi}}
  \def\5{\typeout{current if branch (l.\number\inputlineno):
    \ifnum\3>0 true\else \ifnum\3<0 false\else
      \ifnum0=\3 no\else !UNKNOWN\fi\fi\fi\space branch}}
  \def\6#1{\string\if \ifcase#1!NONE\or \or cat\or num\or dim\or odd\or
    vmode\or hmode\or mmode\or inner\or void\or hbox\or vbox\or x\or
    eof\or true\or false\or case\or defined\or csname\or fontchar\else
    !UNKNOWN\fi}
  \1 0 0
  \if00 \1 1 1 \fi \unless\if00 \else \1 -1 -1 \fi
  \if0a \else \1 1 -1 \fi \unless\if0a \1 -1 1 \fi
  \ifcat00 \1 2 1 \fi \unless\ifcat00 \else \1 -2 -1 \fi
  \ifcat0a \else \1 2 -1 \fi \unless\ifcat0a \1 -2 1 \fi
  \ifnum1=1 \1 3 1 \fi \unless\ifnum1<1 \1 -3 1 \fi
  \ifdim1pt=2pt \else \1 4 -1 \fi \unless\ifdim1pt>2pt \1 -4 1 \fi
  \ifodd5 \1 5 1 \fi \unless\ifodd5 \else \1 -5 -1 \fi
  \setbox2=\vbox{
    \ifvmode \1 6 1 \fi \unless\ifvmode \else \1 -6 -1 \fi
    } \setbox0=\box2
  \setbox1=\hbox{
    \ifhmode \1 7 1 \fi \unless\ifhmode \else \1 -7 -1 \fi
    $
      \ifmmode \1 8 1 \fi \unless\ifmmode \else \1 -8 -1 \fi
      $
    \ifinner \1 9 1 \fi
    }
  \unless\ifinner \1 -9 1 \fi
  \ifvoid2 \1 10 1 \fi \unless\ifvoid0 \1 -10 1 \fi
  \ifhbox1 \1 11 1 \fi \unless\ifhbox2 \1 -11 1 \fi
  \ifvbox0 \1 12 1 \fi \unless\ifvbox1 \1 -12 1 \fi
  \ifx\abc\relax \else \1 13 -1 \fi \unless\ifx\relax\abc \1 -13 1 \fi
  \ifeof7 \1 14 1 \fi \unless\ifeof5 \else \1 -14 -1 \fi
  \iftrue \1 15 1 \fi \unless\iftrue \else \1 -15 -1 \fi
  \iffalse \else \1 16 -1 \fi \unless\iffalse \1 -16 1 \fi
  \ifcase2 \or \or \1 17 1 \fi \ifcase2 \or \else \1 17 -1 \fi
  \ifdefined\hbox \1 18 1 \fi \unless\ifdefined\abc \1 -18 1 \fi
  \ifcsname hbox\endcsname \1 19 1 \fi
  \unless\ifcsname abc\endcsname \1 -19 1 \fi
  \iffontchar\trip`b \1 20 1 \fi \unless\iffontchar\trip`c \1 -20 1 \fi
%
  \if00 \unless\if00 \else \if0a \else \unless\if0a
   \ifcat00 \unless\ifcat00 \else \ifcat0a \else \unless\ifcat0a
    \ifnum1=1 \unless\ifnum1<1 \ifdim1pt=2pt \else \unless\ifdim1pt>2pt
     \ifodd5 \unless\ifodd5 \else
      \setbox3=\vbox{\ifvmode \unless\ifvmode \else
       \hbox{\ifhmode \unless\ifhmode \else
        $\ifmmode \unless\ifmmode \else \ifinner \unless\ifinner \else
         \ifvoid2 \unless\ifvoid0 \ifhbox1 \unless\ifhbox2
          \ifvbox0 \unless\ifvbox1 \ifeof7 \unless\ifeof5 \else
           \ifx\abc\relax \else \unless\ifx\relax\abc
            \iftrue \unless\iftrue \else \iffalse \else \unless\iffalse
             \ifcase2 \or \or \ifcase2 \or \else
              \ifdefined\hbox \unless\ifdefined\abc
               \begingroup \tracingifs=1
                \ifcsname hbox\endcsname \unless\ifcsname abc\endcsname
                 \iffontchar\trip`b \unless\iffontchar\trip`c
                  \showifs % <== show 44 nested ifs
                 \fi \fi
                \fi \fi
               \endgroup
              \fi \fi
             \fi \fi
            \fi \fi \fi \fi
           \fi \fi
          \fi \fi \fi \fi
         \fi \fi \fi \fi
        \fi \fi \fi \fi $
       \fi \fi }
      \fi \fi }
     \fi \fi
    \fi \fi \fi \fi
   \fi \fi \fi \fi
  \fi \fi \fi \fi
\endgroup

%
% -- Check \iffontchar, \fontcharwd, etc.
\typeout{Checking \string\iffontchar, \string\fontcharwd, etc.:}
\begingroup
  \iffontchar \else \fi    % <== missing font identifier and number
  \iffontchar\textfont2 -1 \else \fi    % <== bad character code
  \iffontchar\font 256 \else \fi    % <== bad character code
  \fontcharwd \fontcharht \fontchardp \fontcharic    % <== can't use
  \def\1#1#2{%
    \def\2##1##2{\ifdim\csname fontchar##1##2\endcsname#1`#2=0pt \else
      \space##1##2=\the\csname fontchar##1##2\endcsname#1`#2 \fi}%
    \typeout{Font \string#1 character #2%
      \iffontchar#1`#2:\2wd\2ht\2dp\2ic\else \space does not exist\fi}}
  \1\font A \1{\textfont1}B \1{\scriptfont2}B \1{\scriptscriptfont3}B
  \1\trip 7 \1\trip D \1\nullfont D
%
  \def\1#1{\cndx=\zero \ctmp=\zero
    \def\2{\iffontchar#1\cndx \advance\ctmp by\one \fi
      \advance\cndx by\one
      \ifnum\cndx<256 \expandafter\2\fi}%
    \2%
    \typeout{Font \string#1 has \number\ctmp\space character%
      \ifnum1=\cndx \else s\fi .}}
  \1\font \1\trip \1\nullfont
\endgroup

%
% -- Check \parshapelength, \parshapeindent, and \parshapedimen
\typeout{Checking \string\parshapelength, \string\parshapeindent, and
  \string\parshapedimen:}
\begingroup
  \parshapelength \parshapeindent \parshapedimen    % <== can't use
  \def\1#1 {\edef\2{\2 #1}}
  \let\2=\empty
  \ifdim\parshapeindent-5=0pt \else\1a \fi
  \ifdim\parshapelength-5=0pt \else\1b \fi
  \ifdim\parshapeindent-1=0pt \else\1c \fi
  \ifdim\parshapelength-1=0pt \else\1d \fi
  \ifdim\parshapeindent-0=0pt \else\1e \fi
  \ifdim\parshapelength-0=0pt \else\1f \fi
  \ifdim\parshapeindent 1=0pt \else\1g \fi
  \ifdim\parshapelength 1=0pt \else\1h \fi
  \ifdim\parshapeindent 5=0pt \else\1j \fi
  \ifdim\parshapelength 5=0pt \else\1i \fi
  \ifdim\parshapedimen-5=0pt \else\1k \fi
  \ifdim\parshapedimen-1=0pt \else\1l \fi
  \ifdim\parshapedimen 0=0pt \else\1m \fi
  \ifdim\parshapedimen 1=0pt \else\1n \fi
  \ifdim\parshapedimen 2=0pt \else\1o \fi
  \typeout{Parshape test 1 \ifx\2\empty OK\else error(s):\2\fi}
  \parshape=2 1pt 2pt 3pt 4pt
  \let\2=\empty
  \ifdim\parshapeindent-5=0pt \else\1a \fi
  \ifdim\parshapelength-5=0pt \else\1b \fi
  \ifdim\parshapeindent-1=0pt \else\1c \fi
  \ifdim\parshapelength-1=0pt \else\1d \fi
  \ifdim\parshapeindent-0=0pt \else\1e \fi
  \ifdim\parshapelength-0=0pt \else\1f \fi
  \ifdim\parshapedimen-5=0pt \else\1g \fi
  \ifdim\parshapedimen-1=0pt \else\1h \fi
  \ifdim\parshapedimen 0=0pt \else\1i \fi
  \typeout{Parshape test 2 \ifx\2\empty OK\else error(s):\2\fi}
  \parshape=2 1pt 2pt 3pt 4pt
  \parshape=2 \parshapelength2 \parshapeindent2
              \parshapelength1 \parshapeindent1
  \let\2=\empty
  \ifdim\parshapeindent1=4pt \else\1A \fi
  \ifdim\parshapelength1=3pt \else\1B \fi
  \ifdim\parshapeindent2=2pt \else\1C \fi
  \ifdim\parshapelength2=1pt \else\1D \fi
  \ifdim\parshapeindent3=2pt \else\1E \fi
  \ifdim\parshapelength3=1pt \else\1F \fi
  \ifdim\parshapeindent9=2pt \else\1G \fi
  \ifdim\parshapelength9=1pt \else\1H \fi
  \typeout{Parshape test 3 \ifx\2\empty OK\else error(s):\2\fi}
  \parshape=2 1pt 2pt 3pt 4pt
  \parshape=2 11\parshapedimen4 11\parshapedimen3
              11\parshapedimen2 11\parshapedimen1
  \let\2=\empty
  \ifdim\parshapedimen1=44pt \else\1A \fi
  \ifdim\parshapedimen2=33pt \else\1B \fi
  \ifdim\parshapedimen3=22pt \else\1C \fi
  \ifdim\parshapedimen4=11pt \else\1D \fi
  \ifdim\parshapedimen5=22pt \else\1E \fi
  \ifdim\parshapedimen6=11pt \else\1F \fi
  \ifdim\parshapedimen99=22pt \else\1G \fi
  \ifdim\parshapedimen100=11pt \else\1H \fi
  \typeout{Parshape test 4 \ifx\2\empty OK\else error(s):\2\fi}
\endgroup

%
% -- Check \numexpr, \dimexpr, \glueexpr, and \muexpr
\typeout{Checking \string\numexpr, \string\dimexpr, \string\glueexpr,
  and \string\muexpr:}
\begingroup
  \numexpr \dimexpr \glueexpr \muexpr    % <== can't use
  \let\9=\relax
  \count43=\numexpr ( ( 2 \9           \dimen43=\dimexpr ( 3pt \9
  \skip43=\glueexpr 4pt plus 3fil \9   \muskip43=\muexpr(5muminus1mu)\9
  \def\1#1 {\typeout{\detokenize{#1}=\the#1}}
    \1\numexpr\count43   \1\dimexpr\dimen43
    \1\glueexpr\skip43   \1\muexpr\muskip43

  % Test arithmetic overflow
  \begingroup
    \tracingassigns=5
    \count44=\numexpr"7FFFFFFE+1      \dimen44=\dimexpr"3FFFFFFEsp+1sp
    \count44=\numexpr-"7FFFFFFE-1     \dimen44=\dimexpr-"3FFFFFFEsp-1sp
    \count44=\numexpr"FFFF*"8000      \dimen44=\dimexpr"7FFFsp*"8000
    \skip44="3FFFFFFFsp \advance\skip44by1sp \relax
      \dimen45=\skip44                \dimen45=\dimexpr1sp*\skip44\9
    \dimen44=-"3FFFFFFFsp \advance\dimen44by-1sp
      \dimen45=\skip44                \dimen45=\dimexpr1sp*\skip44\9
    \count44=\numexpr"7FFFFFFE+2\9    \dimen44=\dimexpr"3FFFFFFEsp+2sp\9
    \count44=\numexpr-"7FFFFFFE-2\9   \dimen44=\dimexpr-"3FFFFFFEsp-2sp\9
    \count44=\numexpr"10000*"8000\9   \dimen44=\dimexpr"8000sp*"8000\9
    \count44=\numexpr"10000*-"8000\9  \dimen44=\dimexpr"8000sp*-"8000\9
    \count44=\numexpr-"10000*"8000\9  \dimen44=\dimexpr-"8000sp*"8000\9
    \count44=\numexpr-"10000*-"8000\9 \dimen44=\dimexpr-"8000sp*-"8000\9
    \count44=\numexpr0/0\9            \dimen44=\dimexpr0pt/0\9
    \count44=\numexpr1/0\9            \dimen44=\dimexpr1pt/0\9
    \count44=\numexpr-1/0\9           \dimen44=\dimexpr-1pt/0\9
  \endgroup

  % Test glue arithmetic (add and sub)
  \begingroup
    \skip90=3pt plus 0fill minus 1fil
    \skip91=3pt plus 1fil minus 0fill
    \skip92=2pt plus -1fil minus 1fil
    \tracingassigns=5
    \skip93=\glueexpr\skip90+0pt       \skip93=\glueexpr\skip90+0pt
    \skip93=\glueexpr--\skip90         \skip93=\glueexpr--\skip90
    \skip93=\glueexpr\skip91+0pt
    \skip93=\glueexpr--\skip91         \skip93=\glueexpr--\skip91
    \skip93=\glueexpr\skip92+0pt
    \skip93=\glueexpr--\skip92         \skip93=\glueexpr--\skip92
    \skip93=\glueexpr\skip90-\skip91
    \skip93=\glueexpr\skip91-\skip92
    \skip93=\glueexpr\skip91+\skip92
  \endgroup

  % Test rounding of division (all combinations of signs)
  \def\2#1 {\edef\3{\3 #1}}
  \def\1#1#2#3#4{\let\3=\empty
    \ifnum#4=\numexpr#2/#3\else\2a \fi
    \ifnum#4=-\numexpr-#2/#3\else\2b \fi
    \ifnum-#4=\numexpr#2/-#3\else\2c \fi
    \ifnum#4=\numexpr-#2/-#3\else\2d \fi
    \typeout{Expr quotient rounding #1 \ifx\3\empty OK\else
      error(s):\3\fi}}
  \11{"3FFFFFFF}{"7FFFFFFF}{0}
  \12{"40000000}{"7FFFFFFF}{1}
  \13{"3FFFFFFE}{"7FFFFFFE}{0}
  \14{"3FFFFFFF}{"7FFFFFFE}{1}
  \def\1#1#2#3#4{\let\3=\empty
    \ifnum#4=\numexpr#2/#3\else\2a \fi
    \ifnum#4=-\dimexpr-#2sp/#3\else\2b \fi
    \ifnum-#4=\glueexpr\muexpr#2mu/"10000\9/-#3\else\2c \fi
    \ifnum#4=\muexpr-\dimexpr#2spplus-1muminus-1fil/-#3\else\2d \fi
    \typeout{Expr quotient rounding #1 \ifx\3\empty OK\else
      error(s):\3\fi}}
  \15{32}{5}{6}   % <== three mu_error's (! Incompatible glue units)
  \16{33}{5}{7}   % <== three mu_error's (! Incompatible glue units)
  \17{25}{4}{6}   % <== three mu_error's (! Incompatible glue units)
  \18{26}{4}{7}   % <== three mu_error's (! Incompatible glue units)

  % Test rounding of fractions
  \begingroup
    \def\1#1 #2 #3 #4 #5 {\let\3=\empty
      \ifnum#5=\numexpr#2*#3/#4\else\2a \fi
      \ifnum-#5=\numexpr-#2*#3/#4\else\2b \fi
      \ifnum-#5=\numexpr#2*-#3/#4\else\2c \fi
      \ifnum#5=\numexpr-#2*-#3/#4\else\2d \fi
      \ifnum-#5=\numexpr#2*#3/-#4\else\2e \fi
      \ifnum#5=\numexpr-#2*#3/-#4\else\2f \fi
      \ifnum#5=\numexpr#2*-#3/-#4\else\2g \fi
      \ifnum-#5=\numexpr-#2*-#3/-#4\else\2h \fi
      \4#1}
    \def\4#1{\typeout{Expr fraction rounding #1 \ifx\3\empty OK\else
        error(s):\3\fi}\let\3=\empty}
    \11 "7FFFFFFE "7FFFFFFE "7FFFFFFD "7FFFFFFF
    \12 "7FFFFFFE "7FFFFFFE "7FFFFFFF "7FFFFFFD
    \def\1#1#2 #3 #4 #5 #6 {\let\3=\empty
      \ifnum\numexpr#6=\numexpr(#3)*(#4)/(#5)\else\2#1 \fi
      \ifdim\dimexpr1sp*#6=\dimexpr(#3sp)*(#4)/(#5)\else\2#2 \fi}
    \1ab "3FFFFFFE "7FFFFFFE "7FFFFFFD "3FFFFFFF
    \1cd "3FFFFFFE "7FFFFFFE "7FFFFFFF "3FFFFFFD
    \1ef "1FFFBFFE "20003FFE "1FFFFFFE "1FFFFFFF
    \1gh "1FFFBFFF "20003FFF "1FFFFFFF "20000000
    \1ij "1FFFC000 "20004000 "20000000 "20000000
    \1kl "1FFFC001 "20004001 "20000001 "20000001
    \errorcontextlines=100
    \1mn 1 "7FFFFFFF 2 "40000000
    \43
  \endgroup

  \bgroup
    \skip44=\glueexpr ( \skip43 ) + 3 pt plus 1 fil minus 1 fil l l
    \muskip44=\muexpr (\muskip43)+3muplus1fill
    \dimen44=\dimexpr\skip43+\count43pt
    \dimen44=\dimexpr(\skip43)+(\count43pt)
    \dimen44=\dimexpr\skip43*\count43
    \skip44=\glueexpr\skip43/\count43
    \skip44=\glueexpr\skip43*2/3
  \egroup

  % Test operator precedence
  \bgroup
    \def\1#1#2#3#4{#1#2#3#4=#2#3(#4)\else
      \typeout{expression error (l.\number\inputlineno)}\fi}
    \1\ifnum\numexpr{1+}{2*3}
    \1\ifnum\numexpr{4-}{5*6}
    \1\ifnum\numexpr{7+}{12/4}
    \1\ifnum\numexpr{4-}{6/3}
    \1\ifdim\dimexpr{1pt+}{2pt*3}
    \1\ifdim\dimexpr{4pt-}{5pt*6}
    \1\ifdim\dimexpr{7pt+}{12pt/4}
    \1\ifdim\dimexpr{4pt-}{6pt/3}
    \1\ifdim\glueexpr{7pt+}{12pt/4}
  \egroup

  % Test glue reference count handling
  \shipout\hbox{}
  \bgroup
    \skip43=\glueexpr1ptplus0ptminus0pt
    \skip43=\glueexpr1pt-0pt+0pt
    \skip43=\glueexpr(((\skip43)))
  \egroup
  \shipout\hbox{}
\endgroup

%
% -- Check \mutoglue and \gluetomu
\typeout{Checking \string\mutoglue\space and \string\gluetomu:}
\begingroup
  \mutoglue \gluetomu    % <== can't use
  \skip1=-\mutoglue-\gluetomu9pt \muskip1=-\gluetomu-\mutoglue9mu
  \skip2=\gluetomu\muskip1  % <== two mu_error's (! Incompatible glue units)
  \muskip2=\mutoglue\skip1  % <== two mu_error's (! Incompatible glue units)
  \tracingassigns=1
  \skip1=\mutoglue1muplus-2muminus-3fil
  \muskip1=\gluetomu1ptplus-2ptminus-3fil
  \skip2=\mutoglue-4muplus5fillminus6filll
  \muskip2=\gluetomu-4ptplus5fillminus6filll
  \skip3=-\mutoglue\muskip1
  \muskip3=-\gluetomu\skip1
  \skip4=\mutoglue-\muskip2
  \muskip4=\gluetomu-\skip2

%
% -- Check (mu)glue identity
  \typeout{Checking (mu)glue identity:}
  \skipdef\165535\1=7ptplus0filminus0fill\muskipdef\232766\2=\gluetomu\1
  \tracingassigns=\1
  \1=--\mutoglue--\muexpr(--\gluetomu--\glueexpr(--\1))
  \2=--\gluetomu--\glueexpr(--\mutoglue--\muexpr(--\2))
  \tracingassigns=0

%
% -- Check \gluestretchorder, \glueshrinkorder, \gluestretch, and \glueshrink
  \typeout{Checking \string\gluestretchorder, \string\glueshrinkorder,
    \string\gluestretch, and \string\glueshrink:}
  \gluestretchorder \gluestretch    % <== can't use
  \glueshrinkorder \glueshrink    % <== can't use
  \skip5=1ptminus0fil
  \skip6=1ptplus0fillminus0filll
  \def\2#1{\typeout{wrong glue #1 (l.\number\inputlineno)}}
  \def\1#1#2pt#3#4pt#5 {%
    \ifnum\gluestretchorder#5=#1 \else \2{stretch order}\fi
    \ifdim\gluestretch#5=#2pt \else \2{stretch}\fi
    \ifnum\glueshrinkorder#5=#3 \else \2{shrink order}\fi
    \ifdim\glueshrink#5=#4pt \else \2{shrink}\fi}
  \def\9{\relax}
  \100pt10pt1ptminus0fil
  \100pt10pt\mutoglue1muminus0fil
  \100pt10pt\mutoglue\gluetomu1ptminus0fil
  \100pt10pt\skip5
  \muskip5=\gluetomu\skip5
  \100pt10pt\mutoglue\muskip5
  \100pt10pt\glueexpr\mutoglue\muexpr\gluetomu\skip5\9\9
  \100pt00pt\glueexpr\skip5+0pt\9
  \120pt30pt1ptplus0fillminus0filll
  \120pt30pt\mutoglue1muplus0fillminus0filll
  \120pt30pt\mutoglue\gluetomu1ptplus0fillminus0filll
  \120pt30pt\skip6
  \muskip6=\gluetomu\skip6
  \120pt30pt\mutoglue\muskip6
  \120pt30pt\mutoglue\muexpr\gluetomu\glueexpr\skip6\9\9
  \100pt00pt\glueexpr\skip6+0pt\9
  \10-2pt1-3pt\skip1
  \10-2pt1-3pt\mutoglue\muskip1
  \125pt36pt\skip2
  \125pt36pt\mutoglue\muskip2
\endgroup

%
% -- Check sparse arrays
\typeout{Checking sparse arrays:}
\begingroup
  \def\2#1{#1\1=-1#1\1=65536#1\1=0#1\1=65535\typeout{\meaning\1=\the\1.}}
  \2\countdef % Bad register code (-1) ...and (65536).
  \2\dimendef % Bad register code (-1) ...and (65536).
  \2\skipdef % Bad register code (-1) ...and (65536).
  \2\muskipdef % Bad register code (-1) ...and (65536).
  \2\toksdef % Bad register code (-1) ...and (65536).
  \tracingrestores=1
  \count20=5
  \count2000=5
  \dimen21=5pt
  \dimen2100=5pt
  \skip22=5pt\relax \muskip2200=5mu\relax
  \dimendef\8=256\relax \let\9=\8 \let\8=\relax \show\9
\endgroup
\begingroup
  \def\4#1#2#3#4{{\typeout{testing #1 registers ...}%
    \5\2{#1}\5\3{#1def}% define, e.g., \count and \countdef
    \newlinechar=`^^J \tracingrestores=1 {\tracingassigns=1
    \22000=#2} \22001=#3 \3\1=2002 \1=#4
    \typeout{\6\22000=\the\22000,^^J \6\22001=\the\22001,^^J
       \6\22002=\the\22002,^^J \7\1=\the\1.}%
    \expandafter\unless\expandafter\ifx\2\toks
% more tests for \count, \dimen, \skip, and \muskip
      \advance\22000by#3 \advance\1by#3 \8
      \multiply\22000by10 \multiply\1by10 \8
      \divide\22000by5 \divide\1by5 \8
    \else
% more tests for \toks
      \begingroup
        \tracingassigns=1
        \toks20=#2 \toks21=#3 \toks2100=#4
        \1=\toks20 \1=\toks21
        \toks2200=\toks20 \toks2200=\toks21
        \1=\toks2000 \1=\toks2001
        \toks2200=\toks2000 \toks2200=\toks2001
        \toks30=\1 \toks30=\toks2000 \toks30=\toks2001
        \toks3000=\1 \toks3000=\toks2000 \toks3000=\toks2001
      \endgroup
    \fi
    \tracingassigns=1 \global\22002=#3 \1=#4 \global\1=#2
    }}
  \def\5#1#2{\edef#1{\csname#2\endcsname}}
  \def\6{\expandafter\string}
  \def\7{\expandafter\meaning}
  \def\8{\typeout{\6\22000=\the\22000,^^J \6\22002=\the\22002,^^J
        \7\1=\the\1.}}
  \4{count}{0}{5}{7} % test \count, \countdef
  \4{dimen}{0pt}{2.5pt}{3.5pt} % test \dimen, \dimendef
  \4{skip}{0pt}{2.5ptplus1fil}{3.5ptminus1fill} % test \skip, \skipdef
  \4{muskip}{0mu}{2.5muplus1fil}{3.5muminus1fill} % test \muskip, \muskipdef
  \4{toks}{{}}{{a b c}}{{d e f}} % test \toks and \toksdef
\endgroup
\begingroup  \typeout{testing box registers ...}
  \def\2#1{#1\1=-1#1\1=32768#1\1=0#1\1=32767\typeout{\meaning\1=\the\1.}}
  \2\mathchardef % Bad mathchar (-1) ...and (32768).
  \def\2#1{#1\1=-1#1\1=65536#1\1=0#1\1=65535\typeout{\meaning\1=\the\1.}}
  \setbox-1=\copy65536 % Bad register code (-1) ...and (65536).
  \global\setbox65536=\copy-1 % Bad register code (65536) ...and (-1).
  \wd65536=\ht-1 % Bad register code (65536) ...and (-1).
  \showbox-1 % Bad register code (-1).
  \def\2#1{\typeout{\string\box#1=%
    \ifhbox#1 \ifvbox#1 \else\ifvoid#1 \else\string\hbox\fi\fi\fi
    \ifvbox#1 \ifhbox#1 \else\ifvoid#1 \else\string\vbox\fi\fi\fi
    \ifvoid#1 void\else
      \ifdim\ht#1 =0pt \else,ht=\the\ht#1 \fi
      \ifdim\dp#1 =0pt \else,dp=\the\dp#1 \fi
      \ifdim\wd#1 =0pt \else,wd=\the\wd#1 \fi
    \fi.}}
  \def\1#1{\2{3210#1}}
  \tracingrestores=1
    \setbox32101=\hbox to 3pt{%
      \global\setbox32102=\vbox to 5pt{%
        \setbox32103=\vtop to 7pt{%
          \showgroups
        }%
      }%
    }
  \showbox32100
  \showbox32101
  \11\12\13
  \setbox32103=\copy32101 \11\13
  \setbox32104=\box32102  \12\14
  \ht32101=2pt \11
  \ht32102=4pt \12
  \dp32103=6pt \13
  \wd32104=8pt \14
  \setbox32105=\hbox{\vbox to1pt{}\hskip5pt} \15
  \setbox32106=\hbox{%
    \unhcopy32105 \15%
    \unhbox32105 \15%
    \unhcopy32105 \unhbox32105 } \15 \16
  \setbox32105=\vbox{\hbox to1pt{}\vskip5pt} \15
  \setbox32106=\vbox{%
    \unvcopy32105 \15%
    \unvbox32105 \15%
    \unvcopy32105 \unvbox32105 } \15 \16
  \setbox32105=\vbox{\vbox to10pt{}\penalty0\vbox to20pt{}} \15
  \setbox32106=\vsplit32105 to 10pt \16 \15
\endgroup

%
% -- Check \lastlinefit
\typeout{Checking \string\lastlinefit:}
\begingroup
  \def\1{\setbox0=\vbox{\noindent\2\2\2\2\3\2}\showbox0
    \setbox0=\vbox{\unvbox0 \setbox0=\lastbox \showbox0 }}
  \def\2{\hbox to30pt{}\hskip 5pt plus 20pt minus 4pt }
  \def\3{}
  \tracingparagraphs=1 \showboxdepth=1
  \hbadness=100 \pretolerance=9000
  \parfillskip=0pt plus 1fill \relax
  \hsize=96pt
    \lastlinefit=-1 \1
    \lastlinefit=500 \1
    \lastlinefit=1001 \1
  \hsize=98pt
    \begingroup
      \rightskip=0pt plus 1fill \relax
      \lastlinefit=1000 \1
      \leftskip=0pt plus -1fill \relax
      \lastlinefit=500 \1
    \endgroup
    \lastlinefit=1000 \1
  \hsize=100pt
    \lastlinefit=0 \1
    \lastlinefit=500 \1
    \lastlinefit=1000 \1
  \hsize=110pt
    \begingroup
      \let\3=\hfil
      \lastlinefit=1000 \1
    \endgroup
    \lastlinefit=500 \1
    \lastlinefit=1000 \1
  \hsize=120pt
    \lastlinefit=0 \1
    \lastlinefit=500 \1
    \lastlinefit=1000 \1
\endgroup

%
% -- Check expansion of V 2 \protected macros
\typeout{Checking expansion of V 2 \string\protected\space macros:}
\begingroup
  \protected\def\1{\omit} \def\2{\omit} \def\3{relax}
  \setbox0=\vbox{\halign{&\typeout{# (l.\number\inputlineno)}\cr
    \1& \1\cr
    \2& \2\cr
    \3& \3\cr}}
\endgroup

%
% -- Check hyphenation and \savinghyphcodes
\typeout{Checking hyphenation and \string\savinghyphcodes:}
\begingroup
  \def\2#1#2 {\language=#1 #2 }
  \def\1#1 {\noindent\trip\ \20#1 \21#1 \22#1 \23#1 \24#1 \par}
  \parfillskip=0pt \hbadness=0 \showboxdepth=0
  \hsize=16383.99999pt \pretolerance=-1 \tolerance=-1
  \setbox0=\vbox{
    \lccode`A=`a \lccode`B=`b
    \language=0 \hyphenation{qq-App qqB-pp}
    \language=1 \hyphenation{qq-App qqB-pp}
    \language=2 \hyphenation{qq-A-pp qqB-pp}
    \language=3 \hyphenation{qq-App qq-B-pp}
    \1ppAqq \1upAqq \1ppBqq \1upBqq
    \lccode`A=`r \lccode`B=`b
    \1ppAqq \1upAqq \1ppBqq \1upBqq
    \lccode`A=`a \lccode`B=`r
    \1ppAqq \1upAqq \1ppBqq \1upBqq
  }
  \setbox0=\vbox{\language=3 \1qqapp \1qqbpp \1qqrpp }
\endgroup

%
% -- Check \savingvdiscards, \pagediscards, and \splitdiscards
\typeout{Checking \string\savingvdiscards, \string\pagediscards, and
  \string\splitdiscards:}
\begingroup
  \setbox27=\vbox{\noindent$\splitdiscards\noindent$\pagediscards}
  \showbox27
  \setbox 27 = \vbox {
    \vbox to 20 pt {}
    \prevdepth = -10000 pt
    \openout 0 = abc
    \penalty 0
    \write 1 {write 1}
    \vskip 10 pt plus 1 pt minus 1 pt
    \write 2 {write 2}
    \penalty 0
    \closeout 3
    \kern 10 pt
    \vbox to 20 pt {}
    \vfil
  }
  \def\1 #1 #2 {%
    \savingvdiscards = #1
    \setbox 28 = \copy 27
    {
      \setbox 0 = \vbox {
        \vsplit 28 to #2 pt
        \prevdepth = -10000 pt
        \splitdiscards
        \box 28
      }
      \showbox 0
    }
  }
  \1 -1 20   \1 0 30    \1 1 20    \1 2 30
  \output = {%
    \setbox 0 = \vbox { \pagediscards }
    \showbox 0
    \showbox 255
    \setbox 0 = \box 255
    \global \deadcycles = 0
  }
  \def\1{{\setbox 0 = \vbox { \pagediscards \showlists }}}
  \savingvdiscards = 0 \vfil \kern 20 pt \1 \penalty -1 \1
  \savingvdiscards = 1 \vfill \kern 2 pt \1 \penalty 10 \1
  \kern 5 pt \write 1 {} \penalty 0
  { \savingvdiscards = 0
    \kern 6 pt \write 2 {} \penalty 1
  }
  \kern 7 pt \write 3 {} \hbox {} \penalty -10000
  \1
\endgroup

%
% -- Check \interlinepenalties, \clubpenalties, \widowpenalties,
%  and \displaywidowpenalties
\typeout{Checking \string\interlinepenalties, \string\clubpenalties,
  \string\widowpenalties, and \string\displaywidowpenalties:}
\begingroup
  \begingroup
    \def\2#1{%
      \typeout{\string#1-1=\the#1-1}%
      \typeout{\string#10=\the#10}%
      \typeout{\string#15=\the#15}%
      \typeout{\string#1\string#10=\the#1#10}}
    \def\1{%
      \2\interlinepenalties
      \2\clubpenalties
      \2\widowpenalties
      \2\displaywidowpenalties}
    \def\5#1#2{\number#1#2
      \ifnum#2<#10 \space\expandafter\5\expandafter#1\expandafter
        {\number\numexpr#2+1\expandafter}\fi}
    \def\4#1{\typeout{\string#1=\5#1{0}}}
    \def\3{%
      \4\interlinepenalties
      \4\clubpenalties
      \4\widowpenalties
      \4\displaywidowpenalties}
    \tracingassigns=1 \tracingrestores=1
    \3 \1
    \interlinepenalties=3 101 102 103
    \clubpenalties=1 1 \clubpenalties=4 201 202 203 204
    \widowpenalties=5 301 302 303 304 305
    \displaywidowpenalties=6 401 402 403 404 405 406
    \3 \1
    \setbox0=\vbox{}
  \endgroup
  \parfillskip=0pt \interlinepenalty=7 \clubpenalty=500 \widowpenalty=300
  \displaywidowpenalty=310 \rightskip=0ptplus1fil \hsize=20pt
  \def\3#1 {\setbox0=\lastbox \unskip \count0=\lastpenalty \unpenalty
    \ifnum\count0=#1 \else \typeout{Wrong penalty \number\count0
      \space should be #1 (l.\number\inputlineno)}\fi}
  \def\2{\vrule height1ptwidth19pt\hskip5pt }
  \def\1#1#2 #3 #4 #5 #6 #7 {\setbox0=\vbox{#1\noindent\2\2\2\2$$
      $$\2\2\2\2\par\3#7 \3#6 \3#5 \30 \30 \30 \30 \3#4 \3#3 \3#2 \30 }}

  \1{\interlinepenalties=8 8 7 6 5 4 3 2 1 }508 7 316 501 1 301
  \1{\clubpenalties=2 200 100 }207 107 417 207 107 407
  \1{\widowpenalties=2 2000 1000 \displaywidowpenalties=3 2200 1100 0 }%
    507 1107 2207 1507 1007 2007
\endgroup

%
% -- Check hyphenation of LR and RL segments
\typeout{Checking hyphenation of LR and RL segments:}
\begingroup
  \setbox0=\vbox{\trip
%    \language=5
    \lccode`M=`M \hyphenation{MM-MM}
    \TeXXeTstate=1
    \hsize=0pt
    \parfillskip=0pt
    \noindent{} MMMM \beginL MMMM MMMM MMMM\endL{} \beginR MMMM\endR \par
  }
\endgroup

%
\showboxbreadth=10\showboxdepth=10
\tracingonline=1\tracingoutput=1
\end
\error{e-VirTeX: can't happen in e-trip test!}
%% End of file `etrip.tex'.
