%% (e)uptex or aleph

% Copyright 2025 Japanese TeX Development Community <issue@texjp.org>
% You may freely use, modify and/or distribute this file.

\ifx\kcatcode\undefined\else
% for (e)upTeX
  \kcatcode"C0=14  % Latin-1 Supplement
  \kcatcode"100=14 % Latin Extended-A
  \kcatcode"1E00=14 % Latin Extended Additional
\fi
\ifx\ocp\undefined\else
% for aleph
  \ocp\ORGin=inutf8
  \InputTranslation currentfile \ORGin
\fi
\font\x=eu3-lmr10 \x
\parindent0pt

% Latin-1 Supplement
\catcode"C1=1
\catcode"C2=2
\catcode"C3=3
\catcode"C4=4
\catcode"C5=6
\catcode"C6=7
\catcode"C7=8
\catcode"C8=11
\catcode"C9=12
\immediate\write16{\meaning Á; % begin-group character
                   \meaning Â.}% end-group character
\immediate\write16{\meaning Ã.}% math shift character
\immediate\write16{\meaning Ä.}% alignment tab character
\immediate\write16{\meaning Å.}% macro parameter character
\immediate\write16{\meaning Æ.}% superscript character
\immediate\write16{\meaning Ç.}% subscript character
\immediate\write16{\meaning È.}% the letter
\immediate\write16{\meaning É.}% the characer

% Latin Extended-A
\catcode"101=1
\catcode"102=2
\catcode"103=3
\catcode"104=4
\catcode"105=6
\catcode"106=7
\catcode"107=8
\catcode"108=11
\catcode"109=12
\immediate\write16{\meaning ā; % begin-group character
                   \meaning Ă.}% end-group character
\immediate\write16{\meaning ă.}% math shift character
\immediate\write16{\meaning Ą.}% alignment tab character
\immediate\write16{\meaning ą.}% macro parameter character
\immediate\write16{\meaning Ć.}% superscript character
\immediate\write16{\meaning ć.}% subscript character
\immediate\write16{\meaning Ĉ.}% the letter
\immediate\write16{\meaning ĉ.}% the characer

% Latin Extended Additional
\catcode"1E01=1
\catcode"1E02=2
\catcode"1E03=3
\catcode"1E04=4
\catcode"1E05=6
\catcode"1E06=7
\catcode"1E07=8
\catcode"1E08=11
\catcode"1E09=12
\immediate\write16{\meaning ḁ; % begin-group character
                   \meaning Ḃ.}% end-group character
\immediate\write16{\meaning ḃ.}% math shift character
\immediate\write16{\meaning Ḅ.}% alignment tab character
\immediate\write16{\meaning ḅ.}% macro parameter character
\immediate\write16{\meaning Ḇ.}% superscript character
\immediate\write16{\meaning ḇ.}% subscript character
\immediate\write16{\meaning Ḉ.}% the letter
\immediate\write16{\meaning ḉ.}% the characer


% catcode 1,2 : begin-group, end-group character
\catcode"F2=1  % ò
\catcode"F3=2  % ó
\catcode"102=1 % Ă
\catcode"103=2 % ă
òabcó Ădefă {ghió òjkl} {mnoă Ăpqr}

% catcode 11,12 : the letter, the characer
\catcode"F2=11
\catcode"F3=12
\catcode"102=11
\catcode"103=12
òabcó Ădefă {ghió òjkl} {mnoă Ăpqr}

% catcode 3 : math shift character
\catcode"FC=3  % ü
\catcode"10C=3 % Č
\immediate\write16{\meaning ü.}% math shift character
\immediate\write16{\meaning Č.}% math shift character
% Setting for math fonts is required.
%$d^nx/dt^n$ $d^nx/dt^nü Čd^nx/dt^n$ üd^nx/dt^nČ

% catcode 7,8 : superscript, subscript character
\catcode"FE=7  % þ
\catcode"FF=8  % ÿ
\catcode"10E=7 % Ď
\catcode"10F=8 % ď
\immediate\write16{\meaning þ.}% math shift character
\immediate\write16{\meaning ÿ.}% math shift character
% Setting for math fonts is required.
%$a_nx^m$ $aÿnxþm$ $aďnxĎm$

% catcode 4 : alignment tab character
\catcode"F4=4  % ô
\catcode"104=4 % Ą
\halign{# & # & # \cr
  aaa & bbb & ccc \cr
  ddd ô eee ô fff \cr
  ggg Ą hhh Ą iii \cr}

% catcode 6 : macro parameter character
\catcode"F5=6  % õ
\catcode"105=6 % ą
\catcode"1E00=6 % Ḁ
\def\oWz#1{o{#1}z}\relax
\oWz{r}
\def\oXzõ1{o{õ1}z}\relax
\oXz{r}
\def\oYzą1{o{ą1}z}\relax
\oYz{r}
\def\oZzḀ1{o{Ḁ1}z}\relax
\oZz{r}

\end

