\documentclass[a4paper,english]{article}
\usepackage{a4wide}
\usepackage{latex2man}

\setVersion{1.0}
\setDate{6-9-2024}

\begin{document}
\begin{Name}{1}{texprofile}{Martin Ruckert}{Displaying TeX profiles}{TeXprofile: Displaying TeX profiles}
  
  \Prog{texprofile} - Displaying the profile data collected by \Prog{texprof}.
\end{Name}

\section{Synopsis}

\Prog{texprofile} [Options] \Arg{inputfile}

\section{Description}
\Prog{texprofile} reads the binary \Arg{inputfile} as produced by
\Prog{texprof}.  The extension \File{.tprof} of the input file can be
omitted. It analyses the data and presents the results in a variety of
tables, either optimized for human readability or as CSV (comma
separated values) files for further processing.
The output is written to the standard output stream where it can be redirected
to a file or viewed immediately.

\section{Options}
There are four types of options: general options, table options, selection options, and formatting options.

The general options:
\begin{description}
\item[\Opt{-?} \Opt{-h} \Opt{--help}]
  Display a short help text and exit the program.
\item[\Opt{--version}]
  Display the version information and exit.
\end{description}

The table options determine the tables that will be displayed.
These options all use upper case letters.
If no table option is given, only some global information is shown.

\begin{Description}
\item[\Opt{-T}]
  Show the table of the top 10 input lines.
\item[\Opt{-G}]
  Show the table of the  macro call graph.
\item[\Opt{-C}]    Show the table of times per TeX command.
\item[\Opt{-L}]    Show the table of times per input line.
\item[\Opt{-R}]    Show the table of raw time stamps.
\item[\Opt{-S}]    Show the table of macro stack changes.
\item[\Opt{-F}]    Show the table of all input files.
\item[\Opt{-M}]    Show the table of all macros called.
\item[\Opt{-A}]    Show important tables (equal to -TGFC) tables.
\item[\Opt{-N}]    Do not show the table of global information.
\end{Description}

The selection options allow to reduce the amount of data that is shown
in the table by omitting table entries that contribute little to the
overall runtime.

\begin{description}
\item[\OptArg{-p}{n}] Do not show information for table entries
  with cumulative time below \Arg{n} percent. The default is 1.0 percent.
\item[\OptArg{-t}{n}] Limit the number of input lines shown in the
  table of the ``Top Ten'' to \Arg{n}. This option is only useful
  with the \Opt{-T} option.
\end{description}

The formatting option determine the formatting of the information in the tables
as well as the selection of information that is shown in the tables.

\begin{description}
\item[\Opt{-i}]
  Add the macros file and line numbers after the macro name.
  This option is useful to distinguish two macros that share
  the same name.
\item[\Opt{-m}]
  Optimize output for machine readability. When optimized for human readability,
  times are rounded and given in a short form adding a unit identifier:
  s for seconds, ms for milliseconds, us for micro seconds, and ns for
  nanoseconds. This format is inconvenient if you want to import the data
  for example into a spread-sheet for further processing.
  With the \Opt{-m} option times are simply given in nano seconds without units.
  Similar the combined time for the total time used for a macro and the
  time used as a sub-macro from the call graph is given in a single column
  with a slash separating both numbers is nice for human readers.
  With the \Opt{-m} option both numbers will simply be given in separate columns.
  
\item[\Opt{-n}]
  Show the time stamp numbers. This option is useful together with
  the \Opt{-R} option, if you import the table into a database or spread-sheet.
  If you sort the data by various properties into different
  categories, the time stamp number can be used to keeping the entries
   within a category sorted by time.
  
\item[\Opt{-s}]
  Show the changes of the macro stack.
  This option is useful with the \Opt{-R} option if you want to see the changes
  in the macro stack (see \Opt{-S}) together with the timing information in
  a single table.
\end{description}


\section{See Also}
  \Cmd{texprof}{1}


\section{Version}
Version: \Version\  of \Date
\begin{description}
\item[Copyright] \copyright\ 2024, Martin Ruckert,\\
     Hochschule München, Lothstrasse 64, D-80335 München

\item[Distribution]
  \Prog{texprofile} is distributed with TeX Live.
  More recent versions might be found on GitHub
  \URL{https://github.com/ruckertm/HINT}.

\item[License] This program can be redistributed and/or modified under the
  terms of the MIT/X11 license.
\end{description}

\section{Author}
\noindent
Martin Ruckert                        \\
Munich University of Applied Sciences \\
Email: \Email{martin.ruckert@hm.edu}  \\

\end{document}
